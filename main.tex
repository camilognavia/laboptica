\documentclass{replab}
\usepackage{lipsum}
\usepackage{float,physics,setspace,caption,placeins}
\captionsetup[table]{position=bottom}
% Para colocar pie de tabla es necesario colocar primero el tabular y luego el caption dentro del entorno table.

\onehalfspacing

% --- Informaci\'on del documento ---
\title{Pr\'actica #PRACTICA}

% Nota: Para escribir los nombres de los autores, en el documento replab.cls dentro de la sección "Página de Título" cambiar los valores de nombres y correos dentro de la tabla.

\date{FECHA ENTREGA}
\subtitle={NOMBRE PRÁCTICA}
\subject={Laboratorio de Óptica}

\setlength{\columnsep}{14pt}

% --- Archivo de bibliograf\'ia ---
\addbibresource{repbib.bib}

% --- Inicio del documento ---
\begin{document}
	
	\pagestyle{fancy}
	
% --- T\'itulo ---
		\begin{center}
			\maketitle
			
			{\begin{tcolorbox}[colframe=white, colback=principaldos, arc=8pt]
				\begin{onecolabstract}
                    \vspace{-0.5cm}					
					\medskip
                    
                    \noindent\textit{Palabras clave: }
				\end{onecolabstract}

				\tcblower

				\selectlanguage{english}
				\begin{onecolabstract}
                    \vspace{-0.5cm}						
					\medskip
                    
					\noindent\textit{Keywords: }
				\end{onecolabstract}
			\end{tcolorbox}}

			\smallskip
		\end{center}

	\saythanks

	\selectlanguage{spanish}
	
% --- Cuerpo del reporte ---
	
	\section{Introducci\'on}       
        
    \section{Metodología Experimental}

    \section{Resultados y Discusión}
        
	\section{Conclusiones}
        
\printbibliography[heading=bibintoc]

		
\end{document}
